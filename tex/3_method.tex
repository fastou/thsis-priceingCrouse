\chapter{模型构建}
\label{chapter:模型构建}

上一章系统梳理了教育产品个性化定价的理论基础,明确了用户行为数据在推动教育产品价值识别与差异化定价中的关键作用。这些理论框架为建立面向教育场景的动态定价策略提供了方法指导。

本研究聚焦于G公司初三数学教育产品的差异化组合定价策略设计。与一般标准化教育产品不同,初三数学知识体系结构复杂,不同学生在知识点掌握程度、学习行为习惯以及支付能力等方面存在显著差异。同时,平台可提供的教辅服务模块(如视频讲解、真题练习、真人答疑、动态评测、学习系统使用时长)具备高度组合灵活性,产品定价空间较大。如何基于用户差异化特征,动态生成最优知识点-服务组合,并实现收益最大化,成为平台差异化定价策略设计的重要课题。

针对上述业务需求,本文设计了以用户行为分群、知识点掌握状态建模、支付意愿建模与差异化组合定价策略为核心环节的建模流程,形成端到端差异化组合定价优化框架,为平台个性化定价策略提供理论与方法支持。

本章后续将依次展开用户行为分群建模、知识点掌握状态建模、支付意愿建模与差异化组合定价策略模块的详细设计与建模过程描述。
\section{用户行为分群建模}
\label{sec:用户行为分群建模}

在初三数学教育产品定价实践中,平台用户在学习行为习惯、活跃度、稳定性及学习过程中的反馈表现等方面存在显著差异。若不区分用户行为差异,采用统一策略设计组合产品及定价,容易导致策略匹配度不足,影响整体收益表现及用户体验。因此,有必要在差异化组合定价策略过程中引入用户行为分群建模模块,基于用户行为特征,划分差异化用户群体,支撑后续策略空间约束及个性化组合推荐逻辑设计。

\subsection{分群建模目标}

本模块旨在通过学习行为特征分析,识别用户整体行为模式,生成用户分群标签 $g_i$,作为差异化组合定价策略过程中的关键输入变量之一,主要发挥以下作用:

\begin{itemize}
\item \textbf{策略空间约束}:根据用户行为分群结果,动态调整推荐组合复杂度及价格区间,提升策略匹配度;
\item \textbf{差异化策略引导}:不同用户群体关注点、学习时间有不同的差异,支付意愿差异明显,通过分群结果引导个性化策略设计;
\item \textbf{模型输入特征增强}:将分群标签作为支付意愿和定价策略建模输入特征之一,增强模型预测能力。
\end{itemize}

\subsection{行为特征设计}

为增强用户行为分群模型的业务解释性与应用价值,本文选取以下关键行为特征作为模型输入,形成用户学习行为特征向量 $x_j$:
\begin{itemize}
\item \textbf{avg\_daily\_learning\_time}:用户日均学习时长;
\item \textbf{skip\_rate}:跳题率,反映用户学习稳定性;
\item \textbf{pause\_count}:平均停顿次数,衡量思维犹豫程度;
\item \textbf{stroke\_count}:平均笔画数,反映操作复杂度;
\item \textbf{writing\_variability}:书写速度标准差,体现节奏稳定性;
\item \textbf{pen\_pressure\_avg}:书写压力均值,间接反映信心或焦虑状态。
\end{itemize}
上述特征均通过智能学习硬件与平台学习数据实时采集与处理,具有较高的数据质量与业务可解释性。行为特征向量 $x_j$ 将作为分群建模过程中的核心输入变量。

\subsection{分群建模方法}

本研究借鉴Sun and Chen(2021)在MOOC平台用户分群建模研究成果,采用K-Means聚类方法实现用户行为分群建模,模型具有简洁性、可解释性强、收敛快速等特点,适用于当前业务场景下高维连续行为特征空间建模需求。

K-Means模型目标为最小化簇内平方误差(Within-Cluster Sum of Squares, WCSS):

\begin{equation}
\mathcal{L}_{\text{KMeans}} = \sum_{i=1}^{K} \sum_{x_j \in C_i} \| x_j - \mu_i \|^2
\end{equation}

其中:

\begin{itemize}
\item $x_j$ 为用户 $j$ 的学习行为特征向量;
\item $\mu_i$ 为簇 $C_i$ 的质心向量;
\item $K$ 为预设聚类簇数;
\item $C_i$ 为第 $i$ 个簇对应的用户集合。
\end{itemize}

\subsection{分群建模结果}

分群建模结果以用户分群标签 $g_i$ 形式输入后续模块,标签取值为 $g_i \in \{A, B, C, D\}$,具体含义如下:

\begin{itemize}
  \item \textbf{A类—高活跃型}:日均学习时长长,跳题率低,行为频繁且持续稳定,反映出较高的学习投入与平台黏性;
  
  \item \textbf{B类—波动型}:学习行为不稳定,跳题率与停顿率较高,书写节奏波动大,表现出阶段性参与与任务执行不连贯的特征;
  
  \item \textbf{C类—低活跃型}:学习频次低,跳题率高,学习投入不足,行为特征稀疏,反映出低使用意愿或平台依赖度;
  
  \item \textbf{D类—高效率型}:操作节奏稳定、书写特征一致性强、平均答题时间短,虽整体学习时间不一定最长,但在单位时间内展现出较强的任务完成效率。
\end{itemize}


后续支付意愿建模模块与差异化组合定价策略模块将基于该分群标签 $g_i$ 引导策略空间设计,提升个性化策略匹配度与商业实用性。

\subsection*{小结}

本节围绕用户学习行为分群建模展开,基于K-Means聚类方法,利用用户学习过程中的关键行为特征,构建用户分群标签 $g_i$,为后续支付意愿建模与差异化组合定价策略提供重要输入特征,支撑个性化组合策略设计与收益优化目标实现。

\section{知识点掌握状态建模}
\label{sec:知识点掌握状态建模}

教育产品差异化组合定价策略设计需充分考虑用户在不同知识点层面的掌握差异,动态推荐针对性组合产品,提升定价策略匹配度与用户感知价值。因此,在差异化组合定价优化流程中,需构建知识点掌握状态建模模块,基于用户学习行为数据,判定用户当前阶段薄弱知识点集合 $Z_i$,支撑后续组合策略设计。

\subsection{建模目标}

本模块目标为基于用户学习行为特征,针对知识点粒度,计算用户对各知识点的掌握度得分 $M_{ij}$,并依据设定阈值判定薄弱知识点集合 $Z_i$,作为差异化组合定价策略模块的知识点输入变量。

\subsection{模型设计思路与引入}

在初三数学产品场景中,用户在同一知识点 $z_j$ 上的学习行为具有高度个体差异性:

\begin{itemize}
\item 用户在 $z_j$ 上完成的题目数量存在差异,且通常不会覆盖该知识点下所有题目;
\item 不同题目对知识点掌握度的贡献不同,核心题目(如高频考点题、典型解法题)对掌握度贡献更大,非核心题目影响相对较小;
\item 不同知识点对行为特征的敏感度不同,难度高的知识点(如数列)中长时间作答可能正常,简单知识点(如一元二次方程)中快速完成反映掌握更好。
\end{itemize}

若直接采用用户整体行为特征加权计算掌握度,容易忽略上述因素,导致掌握度判别不准确,影响后续组合策略设计。因此,本文设计分层建模策略:

\begin{itemize}
\item 在题目粒度 $q$ 上,融合用户在该题目上的行为特征,计算题目掌握评分 $S_{iqj}$;
\item 在知识点粒度 $z_j$ 上,按题目权重 $w_q$ 对题目掌握评分进行加权平均,计算用户对知识点的总体掌握度 $M_{ij}$。
\end{itemize}

例如,若用户在知识点“数列”上做了 $5$ 道题目,其中包含 $3$ 道核心题,$2$ 道非核心题,则该用户对“数列”的掌握度 $M_{ij}$ 应主要由核心题的掌握情况决定,非核心题贡献较小。同时,数列作为高难度知识点,其题目时间类特征在掌握度中权重应降低,以更真实反映掌握情况。

\subsection{行为特征设计}

知识点掌握状态建模需将用户学习过程行为特征映射至题目粒度,本文选取以下特征作为模型输入,特征计算均在题目粒度 $q$ 上进行:

\begin{itemize}
\item $c_{iqj}$:题目 $q$ 上用户 $i$ 的答题正确率(0 或 1);
\item $r_{iqj}$:题目 $q$ 上用户 $i$ 的重做次数(或重做标记);
\item $p_{iqj}$:题目 $q$ 上用户 $i$ 的平均停顿时间;
\item $t_{iqj}$:题目 $q$ 上用户 $i$ 的平均做题时间;
\item $s_{iqj}$:题目 $q$ 上用户 $i$ 是否跳题(跳题记为 1,完成记为 0)。
\end{itemize}

上述特征通过平台用户学习过程行为日志与智能学习硬件数据实时采集,具有良好的数据质量与业务解释性。

\subsection{题目掌握评分模型}

题目掌握评分 $S_{iqj}$ 采用加权Sigmoid融合模型,公式如下:

\begin{equation}
S_{iqj} = \sigma \left( \beta_1^{(z_j)} c_{iqj} + \beta_2^{(z_j)} r_{iqj} + \beta_3^{(z_j)} p_{iqj} + \beta_4^{(z_j)} t_{iqj} + \beta_5^{(z_j)} s_{iqj} \right)
\end{equation}

其中:

\begin{itemize}
\item $\beta_k^{(z_j)}$ 表示知识点 $z_j$ 对特征 $k$ 的敏感度系数,$k = 1, \ldots, 5$;
\item $\sigma(\cdot)$ 为Sigmoid归一化函数,$\sigma(x) = \frac{1}{1 + e^{-x}}$,将评分归一化至 $[0,1]$ 区间;
\item 评分 $S_{iqj}$ 反映用户在题目 $q$ 上的掌握水平,高分表示掌握良好,低分表示掌握薄弱。
\end{itemize}

\subsection{知识点掌握度计算}

用户 $i$ 对知识点 $z_j$ 的掌握度得分 $M_{ij}$ 计算公式如下:

\begin{equation}
M_{ij} = \frac{ \sum\limits_{q \in Q_{ij}} w_q \cdot S_{iqj} }{ \sum\limits_{q \in Q_{ij}} w_q }
\end{equation}

其中:

\begin{itemize}
\item $Q_{ij}$ 为用户 $i$ 在知识点 $z_j$ 下做过的题目集合;
\item $w_q$ 为题目 $q$ 的权重,核心题权重大,非核心题权重低,可依据知识点知识图谱和题目标签设定;
\item $S_{iqj}$ 为用户 $i$ 在题目 $q$ 上的掌握评分。
\end{itemize}

该设计可有效融合用户在知识点 $z_j$ 上不同题目的学习情况,动态反映掌握状态,解决了用户覆盖度不完整、题目价值差异、知识点行为敏感度差异等问题。

\subsection{薄弱知识点判别}

根据掌握度得分 $M_{ij}$,设定判别阈值 $\theta$,判定用户 $i$ 的薄弱知识点集合 $Z_i$:

\begin{equation}
Z_i = \{ z_j \mid M_{ij} < \theta \}
\end{equation}

其中 $\theta$ 可通过平台历史学习效果数据与专家经验共同设定。通常可设置为全平台 $M_{ij}$ 分布下的特定分位点(如25\%分位点),动态适应整体学习效果分布变化。

\subsection{建模结果应用}

薄弱知识点集合 $Z_i$ 将作为差异化组合定价策略模块的知识点输入,支撑个性化组合设计,提升策略匹配度与用户感知价值。后续差异化组合定价策略模块将基于 $Z_i$ 动态推荐知识点-服务组合,优化整体收益表现。

\subsection*{小结}

本节围绕知识点掌握状态建模展开,针对初三数学产品场景中用户学习行为的复杂性,设计了分层建模策略,先在题目粒度计算题目掌握评分 $S_{iqj}$,再在知识点粒度计算掌握度得分 $M_{ij}$,并判定薄弱知识点集合 $Z_i$,为后续差异化组合定价策略提供关键输入变量,增强策略个性化与匹配度。


\section{支付意愿建模}
\label{sec:支付意愿建模}

在差异化组合定价策略设计中,支付意愿建模是实现“行为—价值—价格”映射关系的关键环节。尽管用户行为分群模型已刻画出用户在学习行为与操作习惯上的差异性,但对于产品定价而言,更关键的是识别用户的付费能力与付费意愿,即预测用户在面对不同产品组合与价格区间时的潜在响应水平。

\subsection{建模目标与业务背景}

由于平台用户均已完成至少一次付费操作,传统意义上的“是否付费”标签不再适用。因此,本文关注的是用户在历史订单中表现出的\textbf{付费水平差异},并以此为基础,建立用户“支付意愿强度”预测模型。该模型的目标不仅是判断用户“是否支付”,而是量化其未来在个性化组合推荐中接受价格与服务强度的倾向性,用于后续差异化策略空间设计与收益最大化路径规划。

\subsection{输入特征设计}

本节模型主要依赖两个信息源:

\begin{itemize}
  \item \textbf{支付行为特征 $a_i$}:
  \begin{itemize}
    \item $a_i^{(1)}$:首次付费金额,反映初始支付门槛;
    \item $a_i^{\max}$:最大历史单笔付费金额,反映用户支付能力上限;
    \item $\bar{a}_i$:历史订单均值,反映总体支付预期;
    \item $o_i$:历史订单数量,反映用户对平台服务接受程度;
  \end{itemize}

  \item \textbf{行为分群标签 $g_i$}:
  \begin{itemize}
    \item A类:高活跃型;
    \item B类:中活跃型;
    \item C类:低活跃型;
    \item D类:高效稳定型。
  \end{itemize}
  分群标签体现了用户在学习行为上的类型归属,为支付意愿提供结构性调节因素。
\end{itemize}

\subsection{建模方法与数学表达}

为保持模型简洁与可解释性,本文采用Logit模型对支付意愿强度进行建模,其结构如下:

\begin{equation}
\Pr(y_i^{pay} = 1 | a_i, g_i) = \frac{1}{1 + \exp(-(\theta^\top a_i + \gamma_{g_i}))}
\end{equation}

其中:
\begin{itemize}
  \item $y_i^{pay}$ 表示用户 $i$ 是否具备高支付意愿(构建为标签或分段等级);
  \item $\theta$ 为支付行为特征的权重向量;
  \item $\gamma_{g_i}$ 为分群标签 $g_i$ 对整体支付意愿的调节偏置(如 One-hot 编码);
  \item 输出为支付概率,可视为未来支付能力的估计值。
\end{itemize}

为防止模型对某一价格段出现过度敏感的预测结果,进一步引入价格弹性正则项,以控制输出波动:

\begin{equation}
\mathcal{L} = -\sum_{i=1}^{n} \left[ y_i^{pay} \log \hat{y}_i + (1 - y_i^{pay}) \log(1 - \hat{y}_i) \right] + \lambda \cdot \left| \frac{\partial \hat{y}_i}{\partial p} \right|^2
\end{equation}

该正则项有助于维持支付预测函数对价格变化的稳定响应,从而提升策略平滑性与商业可接受度。

\subsection{模型结果的应用价值}

支付意愿预测结果 $\Pr(y_i^{pay} = 1)$ 将直接作为后续定价策略模块的控制输入,具体功能包括:

\begin{itemize}
  \item \textbf{推荐强度控制}:意愿高者可推荐更复杂、更高价值服务组合;
  \item \textbf{定价区间规划}:在收益最大化模型中参与最优价格搜索;
  \item \textbf{风险调控}:避免向低支付意愿用户推送高价组合,降低流失率。
\end{itemize}

\subsection*{小结}

本节从用户历史订单数据出发,设计了面向平台已有付费用户群体的支付意愿建模框架。模型依托支付行为特征与行为分群标签构建输入,采用Logit结构完成支付倾向建模,并为后续定价策略中的价格、服务等级与组合复杂度设计提供量化依据,实现策略层面的精准性与商业价值最大化。
\section{差异化组合定价策略}
\label{sec:差异化组合定价策略}

在完成用户行为分群、知识点掌握状态与支付意愿建模的基础上,平台需依据个体用户的差异化学习需求与支付能力,制定具有针对性的知识点-服务-时长推荐组合,并在此基础上设置合理价格,最终实现平台整体收益的最优化。本节将该问题拆解为两个核心子任务:\textbf{(1)个体定价策略构建}与\textbf{(2)平台收益优化模型设计},分别阐述推荐组合的生成逻辑与收益函数构建原理。

\subsection{个体推荐组合与定价机制设计}
\label{sec:sub_定价策略设计}

\subsubsection*{(一)组合策略的构成单位}

本研究中推荐的定价对象并非单一服务或题目,而是用户 $i$ 在其薄弱知识点 $z_j$ 下的定制化服务包组合,其构成单位为三元组 $(z_j, s_k, \tau_{ijk})$,即:

\begin{itemize}
\item $z_j$:上文模型构建的薄弱知识点集合;
\item $v_{jk}$:产品服务贡献值,如产品类型(如 1V1 答疑、视频讲解等)每个单位时长价值;
\item $\tau_{ijk}$:该服务在该知识点下的推荐服务时长(单位为分钟数或课时数),由后续公式动态计算。
\end{itemize}

多个三元组构成完整的用户组合推荐策略,记为:

\begin{equation}
combo_i = \left\{ (z_j, v_{jk}, \tau_{ijk}) \mid z_j \in Z_i, v_{jk} \in \mathcal{V} \right\}
\end{equation}

其中 $Z_i$ 表示用户的薄弱知识点集合,$\mathcal{v}$ 表示平台支持的服务价值贡献集合。

\subsubsection*{(二)服务价值贡献}

平台提供的服务类型($s_k$)按照个性化程度与交付成本分为五个等级,见表~\ref{tab:service_level_summary}。不同服务在不同知识点下的服务价值贡献 $v_{jk}$(即单位时长带来的平台预期收益)是推荐定价建模中的核心输入变量。

\begin{table}[!h]
\centering
\caption{服务等级说明}
\begin{tabular}{c c c}
\toprule
服务等级 & 类型 & 特点说明 \\
\midrule
等级1 & 1V1实时答疑 & 高度定制,实时交互,教学资源密集,单价最高 \\
等级2 & 真人答疑 & 延迟响应,有答复保障,人工参与 \\
等级3 & 视频讲解 & 标准内容,适合通识训练,资源固定 \\
等级4 & AI错题巩固 & 自动任务,高效率,反馈可控 \\
等级5 & AI答疑 & 模板响应,低成本,适合入门问题 \\
\bottomrule
\end{tabular}
\label{tab:service_level_summary}
\end{table}

\vspace{0.5em}
服务价值贡献 $v_{jk}$ 并非直接对应用户价格,而是平台根据历史收益、服务成本与教学难度等因素推算出的单位时长价值潜力指标。该指标用于后续收益计算与推荐策略排序。

平台在设定 $v_{jk}$ 时,综合考虑以下三类因素:

\begin{itemize}
\item \textbf{(1)知识点难度系数 $\alpha_j$}:来源于平台知识图谱中知识点 $z_j$ 的标注难度等级,难度越高,学习资源配置应越丰富;
\item \textbf{(2)服务等级成本系数 $\beta_k$}:体现服务 $s_k$ 在单位时间内的资源投入强度,例如1V1答疑的人力排班成本远高于AI答疑;
\item \textbf{(3)历史业务表现系数 $\gamma_{jk}$}:以该服务在该知识点下的历史用户转化率、满意度、续购率等业务指标进行回归打分。
\end{itemize}

为实现量化建模,平台构建如下加权组合函数用于评估单位价值贡献:

\begin{equation}
v_{jk} = w_1 \cdot \alpha_j + w_2 \cdot \beta_k + w_3 \cdot \gamma_{jk}
\end{equation}

其中:

$w_1, w_2, w_3$ 为人工调参设定的全局权重(满足 $w_1 + w_2 + w_3 = 1$),初始由教育产品专家经验设定,后续结合 A/B 实验自动调整;
$\alpha_j$ 为知识点 $z_j$ 的难度等级(如按1-5标定并归一化);
$\beta_k$ 为服务 $s_k$ 的单位时长交付成本(归一化);
$\gamma_{jk}$ 为历史运营数据中该服务-知识点组合的加权表现值。


\vspace{0.5em}
最终得到如下归一化示意矩阵(示意为比例值):

\begin{table}[!h]
\centering
\caption{知识点-服务等级单位时间价值贡献示意矩阵(归一化比例制)}
\begin{tabular}{c c c c c c}
\toprule
知识点 / 服务等级 & 1V1实时答疑 & 真人答疑 & 视频讲解 & AI错题巩固 & AI答疑 \\
\midrule
数列 & 1.00 & 0.50 & 0.20 & 0.10 & 0.05 \\
几何 & 1.00 & 0.50 & 0.20 & 0.10 & 0.05 \\
二次函数 & 0.80 & 0.40 & 0.20 & 0.10 & 0.07 \\
一次方程 & 0.60 & 0.30 & 0.15 & 0.10 & 0.08 \\
实数根式 & 0.40 & 0.20 & 0.10 & 0.12 & 0.10 \\
\bottomrule
\end{tabular}
\label{tab:value_contribution_matrix}
\end{table}

\vspace{0.5em}
上述矩阵不仅体现了平台对不同教学场景的服务结构评估逻辑,也为后续推荐组合 $combo_i$ 的收益估计与策略排序提供基础依据,是整套定价机制中最核心的策略输入之一。



\subsubsection*{(三)推荐服务时长的计算逻辑}

推荐服务时长 $\tau_{ijk}$ 是三元组 $(z_j, s_k, \tau_{ijk})$ 的核心组成部分,决定了平台为每位用户提供服务的深度。本文采用以下计算公式:

\begin{equation}
\tau_{ijk} = \eta \cdot (1 - M_{ij}) \cdot \kappa(g_i) \cdot \Pr(y_i = 1)
\end{equation}

其中:

\begin{itemize}
\item $M_{ij}$:用户 $i$ 对知识点 $z_j$ 的掌握度;
\item $\kappa(g_i)$:用户行为分群 $g_i$ 对推荐强度的调节系数;
\item $\Pr(y_i = 1)$:用户对教育产品的支付意愿预测概率;
\item $\eta$:全局单位服务时长基准(如30分钟)。
\end{itemize}

该设计实现了以下策略逻辑:
掌握度低 → 推荐更多时间;
用户活跃度低 → 推荐时长适度放大;
支付意愿高 → 推荐配置向上增强,体现转化潜力。

\subsubsection*{(四)个体推荐组合价格估计}

平台在计算个体推荐价格时,引入统一的单位服务时间定价基准 $\phi$(例如 $\phi = 5$ 元/分钟),并结合服务价值贡献 $v_{jk}$ 与推荐时长 $\tau_{ijk}$,计算理论定价贡献 $r_{ijk}$:

\begin{equation}
r_{ijk} = \phi \cdot \tau_{ijk} \cdot v_{jk}
\end{equation}

最终,用户 $i$ 的推荐组合整体定价为:

\begin{equation}
R_i = \sum_{(z_j, s_k) \in combo_i} r_{ijk} = \phi \cdot \sum_{(z_j, s_k)} \tau_{ijk} \cdot v_{jk}
\end{equation}

其中:

$\phi$ 为单位时间定价基准,由平台统一设定;

$v_{jk}$ 为服务 $s_k$ 在知识点 $z_j$ 上的单位价值;

$\tau_{ijk}$ 为推荐时长。

该定价机制可实现价格动态联动:在推荐时长、服务类型或价值贡献变化时自动调整总定价。同时定价结果也具备良好的结构可解释性,有利于后续组合的结构优化与迭代推荐。

\subsubsection*{(五)组合推荐示意表}

为增强理解,以下表格展示了平台基于定价基准 $\phi = 5$(单位:元/分钟),结合用户行为与掌握状态,为用户 $i$ 生成的个性化推荐组合策略:

\begin{table}[!h]
\centering
\caption{用户 $i$ 推荐组合策略示例}
\begin{tabular}{c c c c c}
\toprule
知识点 $z_j$ & 服务类型 $s_k$ & 推荐时长 $\tau_{ijk}$(分钟) & 单位价值 $v_{jk}$ & 推荐价值 $r_{ijk}$(元) \\
\midrule
数列 & 1V1实时答疑 & 60 & 1.00 & 300.00 \\
几何 & 真人答疑 & 45 & 0.50 & 112.50 \\
二次函数 & 视频讲解 & 30 & 0.20 & 30.00 \\
一次方程 & AI错题巩固 & 20 & 0.10 & 10.00 \\
\bottomrule
\end{tabular}
\label{tab:combo_example}
\end{table}

推荐组合的总定价为:

\begin{equation}
R_i = \phi \cdot \sum_{(z_j, s_k)} \tau_{ijk} \cdot v_{jk} = 5 \cdot (60 \cdot 1.00 + 45 \cdot 0.50 + 30 \cdot 0.20 + 20 \cdot 0.10) = 452.5 \ \text{元}
\end{equation}

该定价代表平台依据用户特征所推荐服务组合的理论价值估计,并将作为后续收益函数优化的重要输入变量。

\subsection{平台收益最大化优化模型}
\label{sec:sub_收益最大化}

在获得所有用户推荐组合的理论价值估计 $R_i$ 后,平台需要在价格制定和资源分配约束下,优化整体收益函数 $\Pi$。

\subsubsection*{(一)收益函数构建}

在构建平台收益函数时,需明确每项推荐组合的交付成本构成。由于服务等级差异巨大(如1V1需人工,AI服务几乎为零边际成本),平台为各等级服务定义了单位时长成本比例 $q_k$,以对应其对整体收益的影响。示意如下:

\begin{table}[!h]
\centering
\caption{服务等级单位时长成本比例设定}
\begin{tabular}{c c c}
\toprule
服务等级 & 服务类型 & 成本比例 $q_k$(占定价比例) \\
\midrule
等级1 & 1V1实时答疑 & 40\% \\
等级2 & 真人答疑 & 30\% \\
等级3 & 视频讲解 & 15\% \\
等级4 & AI错题巩固 & 8\% \\
等级5 & AI答疑 & 3\% \\
\bottomrule
\end{tabular}
\label{tab:cost_ratio_table}
\end{table}

基于此,用户 $i$ 的推荐组合交付成本 $C_i$ 可估算为:

\begin{equation}
C_i = \sum_{(z_j, s_k)} r_{ijk} \cdot q_k
\end{equation}

进而,平台整体预期收益函数为:

\begin{equation}
\Pi = \sum_{i=1}^{N} f_{\text{pay}}(a_i, g_i) \cdot (R_i - C_i)
\end{equation}

其中:

$f_{\text{pay}}(a_i, g_i)$ 为上一节支付意愿预测模型输出;

$R_i$ 为平台对用户推荐组合的定价估计;

$C_i$ 为服务交付成本,动态与组合结构、时长和服务等级相关。

该模型设计明确将“收益 = 价格 - 成本”机制引入差异化定价策略中,兼顾服务层次差异与个体推荐弹性,为收益最大化策略提供了结构性支持。


\subsubsection*{(二)模型优化意义}

该收益函数兼顾用户行为特征、知识点掌握状态与支付意愿响应能力,实现以下建模目的:

\begin{itemize}
\item 利用分群标签与掌握状态控制推荐强度,体现用户画像驱动;
\item 利用支付意愿预测结果进行价格容忍区间调整,防止过度推高导致流失;
\item 服务等级与推荐时长共同决定成本和效益,确保组合结构合理。
\end{itemize}

\subsubsection*{(三)策略输出应用}

通过对 $\Pi$ 的最优化求解,平台可得到每一位用户的个性化推荐定价方案与最大收益预估,并进一步用于:

\begin{itemize}
\item 教辅资源调度计划(如教师分配、AI服务调用);
\item 市场营销场景下的优惠券策略与价格推荐;
\item 用户续购、二次销售的策略嵌套设计。
\end{itemize}

\subsection*{小结}

本节围绕“用户服务组合推荐—推荐价格估计—平台收益函数优化”的链式结构,系统构建了一个兼顾用户行为、学习水平与支付能力的智能定价机制。通过组合三元组 $(z_j, v_{jk}, \tau_{ijk})$ 的结构性设计,平台可实现产品定价的精细控制与用户体验的精准匹配,为个性化教育服务在智能化运营层面提供可部署的建模基础。



% \section{差异化组合定价策略}
% \label{sec:差异化组合定价策略}

% 在完成用户行为分群建模、知识点掌握状态建模与支付意愿建模之后,差异化组合定价策略模块以用户薄弱知识点集合 $Z_i$、用户支付意愿预测 $\Pr(y_i=1|x_i,g_i,p)$ 以及分群标签 $g_i$ 为输入,优化生成用户个性化知识点-服务组合推荐策略 $combo_i$,实现收益最大化目标。


% \subsection{优化建模目标}

% 差异化组合定价策略的核心目标是根据用户差异化特征动态生成个性化组合推荐策略 $combo_i$,实现以下优化目标:

% \begin{itemize}
% \item \textbf{收益最大化}:最大化用户在推荐组合下的期望收益;
% \item \textbf{策略匹配度提升}:根据用户知识点掌握状态,精准定位推荐知识点;
% \item \textbf{策略复杂度控制}:根据用户分群标签与支付意愿,动态调整推荐组合复杂度与服务等级,提升策略可接受性与用户体验。
% \end{itemize}

% \subsection{知识点优选}

% 虽然知识点掌握状态建模模块已输出用户薄弱知识点集合 $Z_i$,但在实际组合定价策略中,不宜直接推送全部 $Z_i$,否则会导致推荐内容过重,用户认知负担大,转化效果不佳。因此,需对 $Z_i$ 进行优选,筛选出优先推荐知识点子集 $\hat{Z}_i \subseteq Z_i$,作为最终推荐知识点集合。

% 优选逻辑包括以下因素:

% \begin{itemize}
% \item \textbf{掌握度优先性}:优先推荐掌握度 $M_{ij}$ 较低的知识点;
% \item \textbf{服务价值优先性}:优先推荐对应服务价值贡献较高的知识点;
% \item \textbf{组合复杂度控制}:根据 $\Pr(y_i=1)$ 和 $g_i$ 控制 $\hat{Z}_i$ 的推荐规模,意愿高可推荐更多知识点,意愿低则控制数量。
% \end{itemize}

% 整体可设计知识点优选排序得分:

% \begin{equation}
% score_{ij} = -M_{ij} + \rho \cdot \sum_{k} v_{jk}
% \end{equation}

% 其中 $\rho$ 为价值贡献权重系数,排序后选取Top-N个知识点组成 $\hat{Z}_i$,$N$动态调整:

% \begin{equation}
% N_i = \delta(g_i) \cdot \Pr(y_i=1|x_i,g_i,p)
% \end{equation}

% 其中 $\delta(g_i)$ 为不同分群标签对应的基础知识点推荐规模系数。

% \subsection{服务模块等级匹配设计}

% 根据平台实际业务,服务模块按价值等级可划分为以下五类,从高到低分别为:

% \begin{itemize}
% \item \textbf{等级1:1V1实时答疑};
% \item \textbf{等级2:真人答疑};
% \item \textbf{等级3:视频讲解};
% \item \textbf{等级4:AI错题巩固};
% \item \textbf{等级5:AI答疑}。
% \end{itemize}

% 每个知识点 $z_j$ 下可配置若干服务等级模块,组合策略过程中需根据用户支付意愿 $\Pr(y_i=1)$ 动态调整推荐服务等级:

% \begin{itemize}
% \item 高支付意愿用户:优先推荐高价值服务(等级1-3),可搭配等级4-5服务;
% \item 中等支付意愿用户:以等级3-4服务为主,辅以少量高价值服务;
% \item 低支付意愿用户:以等级4-5轻量服务为主,控制组合成本。
% \end{itemize}

% 整体组合复杂度控制策略为:

% \begin{equation}
% L_i = \kappa(g_i) \cdot \Pr(y_i=1|x_i,g_i,p)
% \end{equation}

% 其中 $L_i$ 为推荐服务等级配置深度,$\kappa(g_i)$ 为不同分群标签对应的基础服务等级配置系数。

% \subsection{教辅服务价值建模}

% 平台为不同知识点-服务等级组合预设价值贡献矩阵 $v_{jk}^{(l)}$,其中 $v_{jk}^{(l)}$ 表示知识点 $z_j$ 下服务等级 $l$ 对应服务模块的价值贡献。

% \textbf{业务定价逻辑说明}:

% 平台在实际业务中,教辅服务价值贡献设计不仅考虑服务本身的价值等级,还需结合知识点难度特性:

% \begin{itemize}
% \item 难度高的知识点(如数列、几何)对教学资源需求更高,用户付费意愿更强,且高阶服务(如1V1、真人答疑)交付成本更高 → 在价值贡献矩阵中体现出更高占比;
% \item 简单知识点(如实数、根式计算)教学资源需求相对较低,AI类服务更具性价比 → 高阶服务价值贡献占比较低,AI服务贡献占比较高;
% \item 平台定价策略综合考虑\textbf{用户价值感知}与\textbf{服务交付成本},使得价值贡献矩阵符合商业逻辑与用户体验。
% \end{itemize}

% 示意表如下(按比例展示,非真实价格):

% \begin{table}[!h]
% \centering
% \caption{知识点-服务等级价值贡献示意表(按比例)}
% \begin{tabular}{c c c c c c}
% \toprule
% 知识点 / 服务等级 & 1V1实时答疑 & 真人答疑 & 视频讲解 & AI错题巩固 & AI答疑 \\
% \midrule
% 数列 & 1.00 & 0.50 & 0.20 & 0.10 & 0.05 \\
% 几何 & 1.00 & 0.50 & 0.20 & 0.10 & 0.05 \\
% 二次函数 & 0.80 & 0.40 & 0.20 & 0.10 & 0.07 \\
% 一次方程应用题 & 0.60 & 0.30 & 0.15 & 0.10 & 0.08 \\
% 实数、根式计算 & 0.40 & 0.20 & 0.10 & 0.12 & 0.10 \\
% \bottomrule
% \end{tabular}
% \label{tab:knowledge_service_value}
% \end{table}

% 上述价值矩阵 $v_{jk}^{(l)}$ 为策略提供价值约束依据,影响组合收益计算与推荐排序。

% \subsection{收益最大化模型}

% 最终,差异化组合定价策略模型目标为收益最大化,优化目标函数如下:

% \begin{equation}
% \max_{combo_i} \ \Pi = \sum_{i=1}^{N} \Pr(y_i=1|x_i,g_i,p) \cdot \left( \sum_{(z_j, s_k^{(l)}) \in combo_i} v_{jk}^{(l)} - c_{combo_i} \right)
% \end{equation}

% 其中:

% \begin{itemize}
% \item $N$ 为平台用户总数;
% \item $combo_i$ 为用户 $i$ 推荐的知识点-服务等级组合;
% \item $v_{jk}^{(l)}$ 为知识点-服务等级价值贡献;
% \item $c_{combo_i}$ 为组合 $combo_i$ 的整体成本;
% \item $\Pr(y_i=1|x_i,g_i,p)$ 为用户 $i$ 对当前组合的支付意愿预测;
% \item $\Pi$ 为平台整体收益目标。
% \end{itemize}

% 优化过程中,需在收益最大化与策略复杂度、用户体验之间实现平衡,确保推荐组合策略具备商业可用性与用户接受度。

% \subsection*{小结}

% 本节围绕差异化组合定价策略展开,基于用户薄弱知识点集合 $Z_i$,筛选优选推荐子集 $\hat{Z}_i$,根据支付意愿 $\Pr(y_i=1)$ 和分群标签 $g_i$ 动态调整服务等级推荐,优化生成用户个性化知识点-服务组合推荐策略 $combo_i$,实现收益最大化目标,为平台差异化定价策略提供核心优化机制。



\section{端到端差异化组合定价流程框架}
\label{sec:端到端组合定价优化流程}

在本章前述模块设计基础上,本文构建了完整的端到端差异化组合定价优化流程框架,整体流程如图 \ref{fig:end2end_framework} 所示。



本节提出的端到端优化流程框架,不仅通过用户行为分群、知识点掌握状态和支付意愿模型有效捕捉用户差异化需求与支付能力,更通过收益最大化优化模型实现精准的个性化知识点-服务组合推荐,具备良好的实用性与创新性。

\begin{figure}[!h]
\centering
\begin{tikzpicture}[node distance=1.7cm, every node/.style={align=center, font=\small}, >=stealth]
    % 样式定义
    \tikzstyle{data} = [rectangle, rounded corners, draw=black, fill=gray!20, minimum width=3.5cm, minimum height=1.2cm]
    \tikzstyle{process} = [rectangle, rounded corners, draw=black, fill=blue!10, minimum width=4cm, minimum height=1.2cm]
    \tikzstyle{arrow} = [thick,->]

    % 节点定义
    \node[data] (input) {用户学习行为数据};
    
    \node[process, below of=input] (feature) {用户行为特征提取};
    
    \node[process, left of=feature, xshift=-3.5cm] (cluster) {用户行为分群模型};
    \node[process, below of=feature, yshift=-1.7cm] (knowledge) {知识点掌握状态建模};
    \node[process, right of=feature, xshift=3.5cm] (paymodel) {支付意愿建模};

    \node[process, below of=knowledge, yshift=-2cm] (strategy) {差异化组合定价策略};
    
    \node[data, below of=strategy, yshift=-2cm] (output) {服务组合定价推荐};

    % 箭头连接
    \draw[arrow] (input) -- (feature);
    \draw[arrow] (feature) -- (cluster);
    \draw[arrow] (feature) -- (knowledge);
    \draw[arrow] (feature) -- (paymodel);
    
    \draw[arrow] (cluster) |- (strategy);
    \draw[arrow] (knowledge) -- (strategy);
    \draw[arrow] (paymodel) |- (strategy);
    
    \draw[arrow] (strategy) -- (output);
\end{tikzpicture}
\caption{端到端差异化组合定价优化流程框架}
\label{fig:end2end_framework}
\end{figure}

具体而言,整体流程包括以下核心环节:

\begin{itemize}
\item \textbf{用户学习行为数据采集与特征提取}:通过智能硬件与学习平台日志实时采集数据,挖掘反映用户学习特点的多维特征;
\item \textbf{用户行为分群模型}:刻画用户群体差异,指导不同群体的策略空间约束;
\item \textbf{知识点掌握状态建模}:定位用户薄弱知识点,明确个性化推荐的核心内容;
\item \textbf{支付意愿建模}:动态识别用户的支付能力与价格敏感度,指导个性化组合的服务等级推荐;
\item \textbf{差异化组合定价策略}:整合以上模块结果,通过收益最大化模型,输出个性化的知识点-服务组合策略,精准匹配用户需求与支付能力。
\end{itemize}

\subsection{端到端流程框架的业务意义}

该端到端流程框架在业务实践中具备以下实际意义:

\begin{itemize}
\item \textbf{平台收益优化}:通过精准匹配用户个性化需求与支付能力,显著提升平台整体收益;
\item \textbf{用户体验提升}:避免了资源错配问题,用户获得更贴合需求的个性化组合服务,增强用户满意度;
\item \textbf{市场竞争力强化}:建立基于用户差异化特征的精准定价机制,帮助企业在激烈的市场竞争环境中赢得差异化竞争优势。
\end{itemize}

\subsection{框架的创新性与优势}

相比于传统统一定价与简单聚类推荐方法,本流程框架的创新点体现在:

\begin{itemize}
\item 提出了从用户行为特征、知识点掌握状态到支付意愿多维综合建模框架,形成“行为—能力—意愿”三维逻辑闭环;
\item 结合知识点-服务模块分层定价的创新实践,提出了分等级服务与支付意愿精准匹配的组合推荐机制;
\item 构建了融合多模型预测与收益优化的端到端闭环流程,确保模型效果可落地、可验证,实用性更强。
\end{itemize}

\subsection{框架的可扩展性与其他应用场景}

本端到端差异化组合定价优化流程框架具备良好的扩展性与通用性,适用于其他类似在线教育场景或数字产品领域,如:

\begin{itemize}
\item K12其他学科(语文、英语、物理、化学)个性化服务组合定价;
\item 成人教育在线培训产品的个性化定价策略;
\item 数字内容付费平台(如视频平台、电子书平台)个性化内容推荐与差异化定价策略;
\item 电商会员订阅制产品的个性化分层定价与服务组合优化。
\end{itemize}

该框架的多场景适用性,体现了其良好的理论通用性与业务扩展能力,为未来进一步拓展到更多业务领域提供了理论支撑与实践指导。

\subsection*{小结}

本节提出的端到端差异化组合定价优化流程框架,以用户学习行为数据为基础,结合用户分群、知识点掌握状态与支付意愿建模模块,实现了精准的差异化知识点-服务组合推荐。框架不仅具备显著的商业应用价值与竞争优势,也体现了明显的创新性与扩展性,为企业在精细化运营与个性化服务时代背景下的策略提供了坚实支撑与业务引领。

\section*{本章小结}

本章围绕个性化教育场景下的差异化组合定价策略问题,构建了一套完整的端到端模型框架,实现了用户行为数据驱动的精准定价和服务组合推荐。本章首先明确了业务动机,即通过差异化、精准化的个性化定价服务策略,解决传统统一定价模式下存在的资源错配与用户需求难以满足的问题,助力教育企业在“双减”政策背景下实现精细化运营转型。

具体而言,本章依次完成了用户行为分群模型、知识点掌握状态建模和支付意愿预测模型的构建,并在此基础上设计了差异化组合定价策略模型。

下一章将围绕本章构建的端到端差异化组合定价优化框架,进一步开展模型实现与实证分析,从实际数据中验证本章提出模型的有效性、准确性和商业价值。
