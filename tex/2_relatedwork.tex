% !TeX root = ../main.tex
% -*- coding: utf-8 -*-

\chapter{相关理论基础}
\label{chapter:相关理论基础}

\section{个性化教育理论基础}
\label{sec:个性化教育理论基础}

在前文绪论部分已指出,随着“以学生为中心”的教育理念持续深化,个性化教育已成为推动教育公平、提高学习效率的重要方向。为了更好地理解本研究中“基于学习行为数据的用户分群与个性化定价策略”的理论基础,有必要首先对个性化教育的定义、理论渊源与技术路径进行系统梳理。

\subsection{个性化教育的定义}
个性化教育(Personalized Education)是指根据学生的兴趣、能力、学习习惯和发展需求,制定差异化的教学目标、内容与路径,以实现“因材施教”的教育目的。根据\cite{oecd2018teachers}的定义,个性化学习强调“在教育内容与教学方式上进行灵活调整,使每一位学生都能获得最大的发展机会”。

在中国教育部发布的《关于深化教育教学改革全面提高义务教育质量的意见》中亦明确指出,要“尊重学生个体差异,推进个性化教学和精准教学实践”,推动从“以教师为中心”向“以学生为中心”转型。

\subsection{个性化教育的理论基础}

个性化教育的基本理念强调以学习者为中心,根据学生的差异化特征提供定制化学习资源与支持。早期相关理论基础主要包括:

\begin{itemize}
\item 学习是个体在原有知识结构基础上主动建构意义的过程,建构主义学习理论由 \cite{piaget1971constructivism} 提出,强调尊重学生个体经验与认知方式差异;
\item 多元智能理论由 \cite{gardner1983frames}提出,指出个体智力表现具有语言、逻辑、空间、音乐、身体、自然、人际、自知等多维结构,教育应据此实施多通道输入与多样化反馈策略。
\end{itemize}

近年来,随着教育数据与智能技术的引入,个性化教育的理论体系逐渐拓展:

\begin{itemize}
\item 精准教学理论(Precision Teaching)由 \cite{kubina2012precision}提出,强调通过行为数据进行教学调整,采集学习过程数据并反馈于教学决策,是现代学习分析的前提;
\item 学习分析框架由 \cite{unesco2022analytics}提出,指出应基于学习者数据,理解与优化学习过程与环境,为个性化教育提供数据支撑;
\item 开放学习生态模型由 \cite{siemens2011openlearning}提出,认为个体在多源数据环境下需要推荐与引导机制支持;
\item 适应性学习路径理论由 \cite{chen2020adaptivepath}提出,基于学生知识图谱、能力水平与偏好模型,构建自适应学习路径系统,体现了个性化教育中“动态推送-实时反馈”的技术逻辑。
\end{itemize}

这些理论的共同特征是将“数据—模型—策略”作为个性化教育运行机制的基础,强调对学习行为的细粒度采集与建模,并以系统性设计支持差异化服务。

\subsection{个性化教育的实现路径}
随着教育信息化的发展,个性化教育逐步从理念转向系统实现。当前主流路径包括:

\begin{itemize}
\item 使用点阵笔、智能平板等终端实时采集书写行为数据,为建模提供基础;
\item 运用聚类分析、分类算法对用户行为建模,构建个体画像;
\item 基于画像推荐资源,实现教学策略的个性匹配。
\end{itemize}

在K12与在线教育等领域,个性化推荐与精准教学已成为实践方向,构成“数据—模型—反馈”的闭环路径。

\subsection*{小结}
个性化教育是本研究提出“用户画像—策略设计”逻辑路径的理论起点。通过回顾其演化、理论依据与技术路径,明确了本研究以用户行为为核心开展策略优化的逻辑基础。

\section{教育产品差异化定价与用户价值挖掘}
\label{sec:教育产品定价与价值建模}

尽管本研究以个性化教育为背景,但其研究核心并不在于教学内容或课程优化,而是聚焦教育科技产品作为市场商品的经济属性,尝试通过行为数据挖掘与策略建模实现差异化定价机制,从而提升资源配置效率与商业响应能力。为此,有必要从商品属性、定价理论与价值识别三个方面展开分析。

\subsection{教育产品的商品属性与定价动因}

教育类数字产品本质上是一种融合了“内容传递—服务支持—数据交互”的复合型数字商品,其使用效果强依赖用户的个体差异。与传统商品不同,教育产品的“价值实现”并非即刻可见,而需在持续使用、反馈优化的动态路径中才能体现,这使得其呈现出显著的“非均质消费”特征\cite{zhang2021educationalproduct}。

在实际运营中,用户的行为频次、学习稳定性、任务完成度等都可能影响产品的价值释放路径与感知体验,因此“统一定价”往往造成资源浪费或用户流失。Li and Wang (2020) 研究发现,平台用户参与路径存在明显的阶段性与波动性,若忽视这种动态异质性,极易导致策略失配和边际效益递减。

此外,随着在线教育平台对订阅制、功能解锁制、学段升级制等定价模型的探索,教育产品逐步转化为一种“基于行为的服务型商品”,价格不再仅反映内容成本或平均效用,而应体现用户实际表现与使用潜力(Yang \& Zhang, 2023)。因此,教育平台的定价机制需要具备“行为响应性”与“价值分层能力”。

\subsection{差异化定价的理论基础}

传统的定价机制主要建立在成本核算与市场均价的基础之上,但这类机制难以在用户个体价值高度分化的教育场景中发挥作用。Pigou (1920) 提出的差异化定价理论为该问题提供了理论支撑,指出若企业能基于用户支付意愿或效用预期进行价格划分,既可提升整体收益,也有助于不同群体获得匹配服务。

在教育产品场景中,差异化定价的本质在于通过行为数据识别用户的需求等级与使用潜力,并以此设定合理价格段位。Sun and Chen (2021) 基于MOOC平台实证发现,用户学习路径与价格弹性之间存在系统性关联,表明行为数据可用于价格分级模型的建立。这也意味着,教育平台可以通过识别“高潜力-高付费意愿”的用户群体,实施动态策略以提升整体转化效率。

\subsection{差异化定价建模路径}

为了将行为数据有效转化为定价依据,必须建立从行为识别、潜力分析到支付建模的完整差异化建模路径。该路径通常包括以下三个核心阶段,如图\ref{fig:user_value_model_flow}:

\begin{figure}[!h]
  \centering
  \begin{tikzpicture}[node distance=2cm, every node/.style={align=center, font=\small}, >=stealth]
    % 样式
    \tikzstyle{stage} = [rectangle, rounded corners, draw=black, fill=blue!10, minimum width=3.8cm, minimum height=1.4cm]
    \tikzstyle{output} = [rectangle, rounded corners, draw=black, fill=gray!20, minimum width=3.5cm, minimum height=1.2cm]
    \tikzstyle{arrow} = [thick,->]

    % 主体阶段节点
    \node[stage] (A) {用户行为建模\\User Segmentation};
    \node[stage, right of=A, xshift=3.5cm] (B) {学习潜力预测\\Learning Potential Prediction};
    \node[stage, right of=B, xshift=3.5cm] (C) {支付意愿建模\\Price Acceptance Modeling};

    % 输出节点
    \node[output, below of=B, yshift=-1.5cm] (D) {用于定价策略优化};

    % 箭头
    \draw[arrow] (A) -- (B);
    \draw[arrow] (B) -- (C);
    \draw[arrow] (B) -- (D);
  \end{tikzpicture}
  \caption{用户价值建模三阶段流程图}
  \label{fig:user_value_model_flow}
\end{figure}




首先,在用户行为建模阶段,需要通过数据聚类与行为特征分析方法对用户进行分层。Xu and Zhao (2019) 指出,用户在访问频率、活跃周期、操作稳定性等维度上呈现出显著异质性,通过K-means或密度聚类等算法可有效划分不同用户画像。这一阶段的目标是完成“用户个体差异识别”。

其次,学习潜力预测是价值建模的关键一环。\cite{he2020predicting} 研究表明,利用监督学习算法(如回归树、神经网络)可对用户的学习完成率、测评得分提升情况进行建模,从而评估其后续资源利用效率和平台贡献潜力。这一阶段聚焦“行为效能预测”,为后续定价提供量化输入。

最后,支付意愿建模环节则关注用户对价格变化的响应程度。\cite{choi2021priceacceptance}利用Logit模型刻画用户在不同价格设定下的选择概率分布,反映其对价格敏感度的异质性。同时,\cite{fader2005customerltv}所提出的生命周期价值模型(LTV)为平台在定价策略设计时提供了基于长期贡献视角的评估框架,有助于制定兼顾当前转化与长期留存的策略组合。

通过上述三阶段模型,教育平台可将用户行为数据逐步转化为价格决策信息,构建既具经济效率又具行为解释力的动态定价机制。

\subsection*{小结}

教育产品的定价问题不仅是市场策略问题,更是对“用户行为—使用价值—支付意愿”关系的建模问题。以差异化定价理论为基础,结合行为分群、效能预测与支付建模方法,平台可实现从“观察行为”到“调节价格”的策略闭环。本研究正是基于此逻辑,尝试构建一套面向真实用户行为数据的定价优化方案,以实现教育资源配置与平台收益的双重优化。
\section{用户分群理论基础}
\label{sec:用户分群理论}

在本研究的差异化组合定价策略设计中,用户分群是构建策略逻辑的重要基础环节。通过学习行为数据识别出在学习节奏、行为稳定性、知识掌握状态和资源使用偏好等方面存在结构性差异的用户群体,有助于提升定价策略的个性化精准度与商业应用效果。因此,有必要从用户分群理论的学理基础、方法依据及其在教育场景中的适配性等方面进行系统梳理。

\subsection{市场细分理论与行为分群基础}

用户分群(User Segmentation)理论最早源于市场营销领域。Kotler 在其《Marketing Management》一书中指出,市场细分的核心在于“在异质性市场中识别出内部同质、外部异质的消费者群体,以实现产品与价格策略的精准匹配”(\cite{kotler2016marketing})。这一理论为后续行为分析与个体价值挖掘提供了理论支撑,其基本假设包括:

\begin{itemize}
  \item 消费者之间存在可识别且有商业价值的系统性差异;
  \item 差异可以通过可量化特征表达;
  \item 充分利用差异有助于优化产品策略与价格策略,提升商业绩效。
\end{itemize}

随着平台型教育产品的兴起,市场细分理论进一步演化为行为分群(Behavioral Segmentation),强调通过用户在平台中的实际行为数据进行建模。与基于用户属性的静态分层不同,行为分群聚焦于用户在学习频率、答题路径、资源偏好等行为特征上的差异性,更适用于教育场景下高频互动与深度行为反馈的数据环境(\cite{xu2019usersegmentation})。

\subsection{用户分群方法概述}

在差异化定价策略设计中,用户分群不仅具备理论价值,在实际建模过程中也具有明确的可操作性。本研究采用 K-Means 聚类方法对用户行为画像特征进行分群建模,该方法具有实现简洁、收敛速度快、可解释性强等优点,广泛应用于教育数据挖掘与个性化推荐领域(\cite{sun2021moocpricing})。

K-Means 聚类算法通过最小化簇内平方误差(Within-Cluster Sum of Squares, WCSS)实现用户群体划分,其优化目标函数如下:

\begin{equation}
\mathcal{L}_{\text{KMeans}} = \sum_{i=1}^{K} \sum_{x_j \in C_i} \| x_j - \mu_i \|^2
\end{equation}

其中,$x_j$ 表示用户 $j$ 的学习行为特征向量,$\mu_i$ 表示第 $i$ 个簇的中心向量,$C_i$ 表示属于第 $i$ 个簇的用户集合。后续章节将对具体的特征变量设计与聚类建模过程进行详细展开。

\subsection{用户分群在差异化组合定价策略中的作用}

在本研究中,用户分群不仅是用户理解和用户画像构建的重要手段,更是支撑差异化组合定价策略设计的核心输入之一。分群结果可作用于以下三个层面:

\begin{itemize}
  \item \textbf{服务推荐策略的复杂度调节}:用户分群结果为不同群体的服务推荐结构和组合复杂度提供约束条件。例如,高活跃高接受意愿群体可推送高价值组合,低活跃群体则控制推荐服务数量和价格水平。
  \item \textbf{支付意愿建模中的调节变量}:分群标签作为行为-支付之间的桥梁,可提升支付意愿预测的准确性。已有研究(\cite{choi2021priceacceptance})表明,在用户价值建模中引入群体特征作为条件变量,有助于提升模型性能。
  \item \textbf{策略评估与分层分析依据}:分群结果为策略验证与收益分析提供分层标准,有助于衡量不同群体在资源接受度、定价接受率与产品转化效果方面的异质性表现。
\end{itemize}

\subsection{教育场景中的分群实践启示}

在教育场景中,学生群体在学习活跃度、行为稳定性、知识掌握路径、服务接受意愿等方面具有天然的异质性,通用定价模式无法精准适配不同用户的学习状态与资源需求。MOOC 平台与中小学智能学习平台的实践表明,通过行为特征构建用户分群,并据此开展差异化教学干预与产品推荐,可显著提升用户体验与平台收益(\cite{sun2021moocpricing})。

本研究借鉴上述实践经验,将用户分群作为差异化组合定价策略的重要组成部分,推动定价策略从传统静态规则向行为驱动的动态个性化机制演进。

\subsection*{小结}

本节从市场细分理论、行为分群方法和教育应用实践三个维度,系统梳理了用户分群理论的研究基础。用户分群作为差异化定价机制中的关键支撑,在服务推荐结构调节、支付意愿建模与策略评估分层中发挥重要作用。为后续知识点掌握状态分析、支付意愿刻画与组合策略设计提供理论基础与结构入口。

\section{知识点掌握状态建模理论}
\label{sec:知识点掌握状态建模理论}

在差异化组合定价策略设计中,知识点掌握状态建模是承接用户学习行为特征与服务推荐决策之间的重要环节。通过精准刻画学生在各知识点层面的掌握程度,能够为差异化服务组合设计提供细粒度依据,提升定价策略的个性化水平与资源分配效率。因此,有必要从知识点掌握状态建模的理论基础、方法体系与教育场景适配性等方面进行系统梳理。

\subsection{理论基础与研究启发}

知识点掌握状态建模(Mastery Modeling)起源于教育测量与学习分析(Learning Analytics)领域。其核心理念是通过学生的过程性学习数据,动态推断学生在各知识点上的掌握状态,为个性化教学、推荐系统及教育评估提供支持(\cite{siemens2011openlearning,chen2020adaptivepath})。

早期的知识掌握建模方法主要基于学习知识图谱(Learning Knowledge Graph, LKG),将学习内容结构化表示,并通过图谱节点的属性建模掌握程度(\cite{chen2020adaptivepath})。随着智能学习平台和在线教育环境的发展,过程性学习行为(如答题轨迹、学习时间、反馈行为等)被广泛应用于知识点掌握状态推断(\cite{sun2021moocpricing})。

此外,认知诊断模型(Cognitive Diagnostic Model, CDM)与贝叶斯知识追踪(Bayesian Knowledge Tracing, BKT)等方法也在个性化学习系统中广泛采用,为动态掌握状态建模提供了强有力的理论支持(\cite{delatorre_minchen_2014,corbett_anderson_1995})。

\subsection{建模逻辑与方法体系}

现有研究表明,知识点掌握状态建模主要包含以下几类建模思路:

\subsubsection{加权评分模型(Mastery Scoring Model)}

加权评分模型是当前智能学习平台中广泛应用的知识点掌握状态建模方法之一。该方法通过加权整合学生在学习过程中的多个行为特征,形成连续型掌握度评分(Mastery Score):

\begin{equation}
M_{ij} = \sum_{k=1}^{K} \alpha_k \cdot F_{ijk}
\end{equation}

其中,$M_{ij}$ 为学生 $i$ 在知识点 $j$ 上的掌握度评分,$F_{ijk}$ 为第 $k$ 个特征的量化指标,$\alpha_k$ 为特征加权系数。该方法具有实现简单、可解释性强的优点,广泛用于MOOC平台与K12智能学习产品(\cite{xu2019usersegmentation,sun2021moocpricing})。

\subsubsection{贝叶斯知识追踪模型(Bayesian Knowledge Tracing, BKT)}

BKT 是经典的概率图模型,广泛用于在线学习系统中动态追踪学生对知识点的掌握状态(\cite{corbett_anderson_1995})。BKT 将学生对某知识点的掌握状态建模为隐变量 $L_t$,并通过学生在不同时间步的答题表现更新掌握概率:

\begin{equation}
P(L_t | \text{history}) = \frac{P(C_t | L_t) P(L_{t-1})}{P(C_t)}
\end{equation}

其中 $P(L_t)$ 为时间 $t$ 对知识点的掌握概率,$C_t$ 为当前答题表现。BKT 可动态更新掌握状态,适用于需要实时反馈的智能教学场景。

\subsubsection{项目反应理论(Item Response Theory, IRT)}

IRT 是教育测量领域的经典模型,通过建立学生能力 $\theta_i$ 与题目难度 $b_j$、区分度 $a_j$ 等参数之间的关系,推断学生知识掌握水平(\cite{lord_1980}):

\begin{equation}
P_{ij} = \frac{1}{1 + \exp[-a_j(\theta_i - b_j)]}
\end{equation}

其中 $P_{ij}$ 为学生 $i$ 正确回答题目 $j$ 的概率,$\theta_i$ 作为学生整体能力指标亦可分解至知识点维度。IRT 为掌握状态建模提供了强有力的统计推断基础,近年来也被广泛集成至在线学习平台。

\subsubsection{学习曲线建模(Learning Curve Modeling)}

学习曲线建模通过追踪学生在特定知识点上的学习进展,拟合其掌握水平随时间变化趋势,典型模型形式为幂律学习曲线(Power Law of Learning):

\begin{equation}
Performance(t) = a \cdot t^{-b} + c
\end{equation}

其中 $t$ 为学习次数,$a, b, c$ 为拟合参数。学习曲线建模强调学习动态过程,适用于个性化推荐与服务优化策略设计(Newell \& Rosenbloom, 1981)。

\subsection{知识点掌握状态对差异化定价策略的作用}

知识点掌握状态建模结果可为差异化组合定价策略设计提供以下支持:

\begin{itemize}
  \item \textbf{服务内容推荐依据}:根据学生对不同知识点的掌握状态,动态推荐适配的服务组合,提升推荐效果与用户体验;
  \item \textbf{组合复杂度调节依据}:根据掌握薄弱程度设计不同复杂度的服务组合,提升资源配置效率;
  \item \textbf{支付意愿模型输入特征}:学生在薄弱知识点上的学习需求强烈,掌握状态作为支付意愿预测模型中的关键特征,有助于优化价格接受度预测效果。
\end{itemize}

已有研究(\cite{sun2021moocpricing,xu2019usersegmentation})指出,知识点掌握状态作为教育场景中的核心用户状态变量,能够有效提升差异化推荐与定价策略的优化能力,是智能教育平台实现高质量商业优化的重要支撑。

\subsection*{小结}

本节围绕知识点掌握状态建模的理论基础、典型建模方法与应用价值进行了系统梳理。掌握状态建模不仅在个性化教学中具有重要作用,也是差异化组合定价策略设计中的关键变量之一。通过引入行为驱动掌握建模、BKT、IRT 等多种方法理论,为后续支付意愿特征分析与组合定价优化提供了坚实的理论基础。


\section{行为定价理论基础}
\label{sec:行为定价理论}

行为定价(Behavior-based Pricing)是数字经济中个性化定价机制的一种核心理论,其强调通过用户在平台上的历史行为、使用轨迹与反馈表现,识别个体支付意愿的差异性,从而构建差异化的定价策略。与传统基于成本加成或均衡市场价格的定价方式不同,行为定价关注用户之间的响应异质性,强调“行为—效用—价格”之间的函数映射关系(\cite{chen2002consumer}),已广泛应用于电商推荐、平台激励与数字商品分层定价等场景。

\subsection{行为定价的理论渊源与经济学基础}

行为定价的理论基础可追溯至微观经济学中关于“价格歧视”(Price Discrimination)的研究框架。\cite{pigou1920economics}首次系统提出一级(完全歧视)、二级(版本歧视)、三级(群体歧视)三种形式,其中一级价格歧视即为企业能够精确了解个体支付意愿,并对不同消费者设定不同价格。

行为定价可视为一级与三级定价的融合路径,其不依赖消费者主观申报或精确身份识别,而是通过分析用户客观行为推断支付倾向,在兼顾隐私与公平性的同时实现收益最优化目标。

进一步地,\cite{ariely2003}提出的“coherent arbitrariness”理论指出,尽管个体支付决策带有一定随机性,但在自身行为基础上呈现出相对一致的支付模式,即行为具有稳定的心理锚定效应。这为基于用户行为构建价格预测模型提供了心理学与行为经济学的理论支持。

\subsection{行为变量与支付意愿的函数映射机制}

行为定价的核心机制在于建立一个从用户行为变量空间 $\mathcal{X}$ 到支付概率空间 $[0, 1]$ 的映射函数,即:

\begin{equation}
\Pr(y_i = 1 | x_i, p) = f(x_i, p)
\end{equation}

其中 $x_i$ 表示用户 $i$ 的历史行为特征向量,$p$ 为价格水平,$y_i \in {0, 1}$ 为是否支付的响应变量。函数 $f(\cdot)$ 可采用概率回归、树模型或神经网络实现,其本质是学习一个“行为—支付”间的响应函数。

在本研究中,平台的收益优化目标可表示为:

\begin{equation}
\max_{p} \ \Pi(p) = \sum_{i=1}^{n} \Pr(y_i = 1 | x_i, p) \cdot (p - c)
\end{equation}

其中 $c$ 为单位产品成本,$\Pi(p)$ 表示平台的期望利润函数。优化策略的关键在于设计合适的 $\Pr(\cdot)$ 函数,使其既具有行为解释力,又具备预测准确性。

\subsection{在教育平台中的实际演化路径}

在教育科技产品中,用户的学习轨迹、任务完成率、答题行为等数据为行为定价提供了良好的输入基础。\cite{zhang2021educationalproduct}指出,课程重读频率与夜间学习行为显著预测用户在付费环节的响应强度。\cite{sun2021moocpricing}通过聚类分析发现,MOOC 学习者可被划分为“稳步进阶型”“高频弃学型”“拖延波动型”等群体,其支付意愿呈现出明显的段差分布。

这一发现推动教育平台采用“滚动式行为定价”策略,即通过实时监测用户行为变化更新定价推荐,形成行为响应性价格体系。这种定价模式特别适用于平台订阅制、功能解锁型产品与定制服务包等教育产品形态。

\subsection{行为定价建模框架与优化函数}

本研究采用逻辑回归模型对用户支付概率进行建模,形式如下:

\begin{equation}
\Pr(y_i = 1 | x_i, p) = \frac{1}{1 + \exp(-(\beta_0 + \beta^\top x_i + \beta_p p))}
\end{equation}

其中,$\beta$ 表示行为特征权重,$\beta_p$ 表示价格敏感系数。该模型可通过最大似然估计获得参数,用于刻画个体行为与支付反应之间的交互效应。

为提升模型稳定性与策略鲁棒性,本文引入价格弹性正则项作为目标函数约束项,用以抑制因价格微调而引发支付概率剧烈波动的问题。最终优化目标函数为:

\begin{equation}
\mathcal{L} = -\sum_{i=1}^{n} \left[ y_i \log \hat{y}_i + (1 - y_i)\log(1 - \hat{y}_i) \right] + \lambda \cdot \left| \frac{\partial \hat{y}_i}{\partial p} \right|^2
\end{equation}

该正则项源自 \cite{choi2021priceacceptance}在电子商务定价中的策略控制方法,可提高模型的泛化能力与定价曲线的平滑性,避免策略波动引发用户流失。

\subsection*{小结}

行为定价理论为用户行为变量与支付意愿函数之间建立了系统映射机制,其本质是利用历史数据推断个体价值,并在定价决策中加以体现。作为本研究中支付建模的理论基础,本节所述方法为后续策略函数设计、最优价格求解与效益评估提供了结构化的理论支持。


\section{支付意愿建模理论}
\label{sec:支付意愿建模理论}

尽管行为定价强调“行为决定价格”,但在实际策略设计中,平台需要一个明确的函数模型将行为变量 $x_i$ 映射为用户对定价 $p$ 的响应概率 $\Pr(y_i=1|x_i,p)$。支付意愿建模(Willingness-to-Pay Modeling)正是完成这一映射的关键机制,决定了平台后续策略的可操作性、可解释性与优化空间。

\subsection{支付意愿的理论起点}

在传统消费者选择理论中,支付意愿被定义为个体对某一产品或服务的“主观边际效用达到价格水平”的阈值点(\cite{lancaster_1966})。随着行为经济学与信息技术的发展,研究者开始强调用户支付意愿的可预测性,即通过个体历史行为、社会属性与互动偏好建模其价格响应特征(\cite{louviere_et_al_2000})。

在平台场景中,支付意愿并非静态存在,而是受用户行为状态与外部价格刺激共同影响的动态函数。因此,建模用户支付概率 $\Pr(y_i = 1|x_i, p)$,不仅是对个体经济理性的刻画,更是平台收益函数构建与策略最优化的基础。

\subsection{概率建模路径:Logit 模型及其机制}

本研究采用逻辑回归(Logistic Regression)作为支付意愿的建模工具,其原因包括:

在预测任务中具有良好的可解释性;

能处理二分类问题;

可拓展到收益函数最优化任务中。

具体模型设定如下:

\begin{equation}
\Pr(y_i = 1 | x_i, p) = \frac{1}{1 + \exp(-(\beta_0 + \beta^\top x_i + \beta_p p))}
\label{eq:logit}
\end{equation}

其中:

$x_i$ 为用户的行为变量;

$p$ 为对应的价格水平;

$\beta$、$\beta_p$ 分别表示行为变量与价格对支付概率的边际影响。

该模型反映了“行为变量 + 价格刺激”共同决定用户支付概率的非线性结构,是行为定价逻辑的量化表达。

\subsection{对策略函数的支持作用}

如前所述,平台的收益函数可定义为:

\begin{equation}
\Pi(p) = \sum_{i=1}^{n} \Pr(y_i = 1 | x_i, p) \cdot (p - c)
\end{equation}

其中,$\Pr(y_i = 1 | x_i, p)$ 由公式 \eqref{eq:logit} 提供估计,$c$ 为产品单位成本。因此,支付意愿建模构成了收益最大化问题的目标函数输入。

更重要的是,通过对 $\beta_p$ 与 $\beta^\top x_i$ 的估计,平台可判断用户对价格的敏感性,进而构造行为响应函数与价格分段函数,实现个性化定价逻辑。

\subsection{正则项控制与模型稳定性}

在实际建模中,用户行为变量可能存在噪声,价格调整可能引发策略不稳定。为此,本研究在逻辑回归损失函数中引入价格敏感性正则项,以平滑支付概率曲线:

\begin{equation}
\mathcal{L} = -\sum_{i=1}^{n} \left[ y_i \log \hat{y}_i + (1 - y_i)\log(1 - \hat{y}_i) \right] + \lambda \cdot \left| \frac{\partial \hat{y}_i}{\partial p} \right|^2
\end{equation}

该正则项通过抑制模型对价格变化的极端响应,控制策略输出的稳定性,避免用户体验恶化与收益波动加剧。在电商场景中也采用类似方法验证其有效性。


\subsection*{小结}

支付意愿建模是连接行为变量与价格响应之间的核心机制,决定了平台能否实现从“观察行为”到“引导支付”的策略闭环。通过建立 $\Pr(y_i=1|x_i,p)$ 函数结构,平台可开展价格敏感性分析、个性化定价策略设计与收益优化仿真,为后续章节策略构建提供坚实建模基础。

\subsection{策略评估的理论基础}
\label{sec:策略评估理论}

个性化定价策略的有效性不仅取决于行为识别与支付意愿建模的准确性,更取决于所提出策略在收益提升、用户接受与模型稳定性方面的综合表现。因此,必须建立一套基于经济理论与管理科学的策略评估框架。

\cite{von_hippel_2005}指出,用户对平台机制的接受度往往基于其对价格调整的感知公平性与响应一致性。\cite{shugan_2004}亦强调,数据驱动策略若无系统评估机制,极易在收益函数最优与用户体验之间发生偏离。因此,策略评估必须同时满足以下三个目标:

\begin{itemize}
  \item \textbf{收益最优化}:平台收益是否提升;
  \item \textbf{用户接受性}:用户是否更倾向于完成支付;
  \item \textbf{输出稳定性}:策略结果在不同场景下是否稳健。
\end{itemize}

为量化以上目标,本文将引入以下核心理论指标作为评估工具,其定义如下:

\begin{itemize}
  \item \textbf{支付率(Purchase Rate)}:在定价策略下用户实际完成支付的比例:
  \begin{equation}
    \text{PurchaseRate} = \frac{1}{n} \sum_{i=1}^{n} \mathbb{I}(y_i = 1)
  \end{equation}
  其中,$\mathbb{I}(\cdot)$ 为指示函数,$y_i=1$ 表示用户 $i$ 成功支付。

  \item \textbf{人均收益(Revenue per User, RPU)}:策略实施后的平均收益水平:
  \begin{equation}
    \text{RPU} = \frac{1}{n} \sum_{i=1}^{n} y_i \cdot (p_i - c)
  \end{equation}
  其中,$p_i$ 为分配给用户 $i$ 的价格,$c$ 为产品边际成本。

  \item \textbf{价格弹性(Price Elasticity)}:反映用户支付概率对价格变动的敏感性:
  \begin{equation}
    \text{Elasticity}_i = \frac{\partial \Pr(y_i = 1 | x_i, p)}{\partial p}
  \end{equation}
  弹性值越大,说明用户对价格波动越敏感,策略鲁棒性越低。

\end{itemize}

上述指标将在后续实证章节中用于对不同定价策略的性能进行量化评估,为最终策略选择与推广提供理论依据与定量支撑。

\section*{本章小结}

本章从理论层面系统构建了“行为驱动的个性化定价”研究框架,为后续模型构建与实证分析奠定了理论基础。首先,回顾了个性化教育的发展背景与理论根源,明确了学习行为数据在教育资源精准配置中的核心作用。其次,从教育产品的经济属性出发,引入差异化定价与用户价值识别理论,界定了本研究的策略优化目标与理论依据。
