% !TeX root = ../main.tex
% -*- coding: utf-8 -*-

% 2020版要求4.8
% 个人简历、在学期间发表的学术论文与研究成果:
% 标题要求同各章标题。文字部分:宋体10.5磅(或五号),英文用Times New Roman字体10.5磅(或五号),固定值行距16磅,段前段后0磅,发表学术论文书写格式同参考文献
% 这一部分要求似乎较宽松,大小和行距或许可以不改。

\chapter*{个人简历、在学期间发表的学术论文与研究成果}

\section*{\leftline{个人简历}}

{
\zihaowu
\setlength{\baselineskip}{16pt}
\setlength{\parskip}{0pt}

xxx,出生于yyyy年mm月dd日。
在20yy年毕业于xx大学XX专业并获得xx士学位。
于20xx年至今在南开大学就读xxx研究生。

}

\section*{\leftline{研究生期间发表论文:}}
% 学术论文研究成果按发表的时间顺序列出
% (已发表的列在前面,已接收待发表的放在后面)
% 格式方便阅读为主可参考百度学术Google学术

% 注:这个要求似乎比参考文献宽松,因此没改。若强求格式,可改为普通文本并参照上一段修改字号行距等格式。

{
\zihaowu
\setlength{\baselineskip}{16pt}
\setlength{\parskip}{0pt}

\begin{itemize}
	\item 周恩来. 周恩来选集[M]. 人民出版社, 1980.
	\item 周恩来. 周恩来外交文选[M]. 中央文献出版社, 1990.
\end{itemize}

}



% 其他成果有可添加
% \section*{\leftline{研究生期间其它成果:}}
% % 研究成果可以是在学期间参加的研究项目、申请的专利或获奖等

% {
% \zihaowu
% \setlength{\baselineskip}{16pt}
% \setlength{\parskip}{0pt}

% \begin{itemize}
% 	\item 
% \end{itemize}

% }
